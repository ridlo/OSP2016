\documentclass[11pt,fleqn, a4paper]{exam}
\usepackage[utf8]{inputenc}

\usepackage[margin=1in]{geometry}
\usepackage{amsmath,amssymb}
\usepackage{gensymb}
\usepackage{multicol}
\usepackage{float}
\usepackage{graphicx}
\usepackage{units,icomma}
\usepackage{hyperref}
\usepackage{enumerate}
\usepackage{wasysym}
\usepackage{multirow}
\usepackage[usenames,dvipsnames]{xcolor}
\usepackage[margin=1.5cm]{caption}
%\everymath{\displaystyle}

\hyphenation{
  chro-no-ampe-ro-met-ric
  ber-dia-me-ter
  de-ngan
  me-nem-pati
  mic-ro-graphs}

\renewcommand{\figurename}{Gambar.}
\renewcommand{\labelitemi}{$-$}


\newcommand{\class}{OLIMPIADE ASTRONOMI}
\newcommand{\term}{Tingkat Provinsi - 2016}
\newcommand{\examnum}{OSP Astronomi 2016}
%\newcommand{\examdate}{11/02/2014}
%\newcommand{\timelimit}{120 Minutes}

\pagestyle{head}
\firstpageheader{}{}{}
\runningheader{\examnum}{}{Halaman \thepage\ dari \numpages}
\runningheadrule


\begin{document}

\noindent
\begin{tabular*}{\textwidth}{l @{\extracolsep{\fill}} r @{\extracolsep{6pt}} l}
\textbf{\class} \\% & \textbf{Name:} & \makebox[2in]{\hrulefill}\\
\textbf{\term}  %&&\\
%\textbf{\examnum} &&\\
%\textbf{\examdate} &&\\
%\textbf{Time Limit: \timelimit} & Teaching Assistant & \makebox[2in]{\hrulefill}
\end{tabular*}\\
\rule[2ex]{\textwidth}{2pt}

\noindent
\begin{tabular}{ll}
\textit{Copyright} (c) 2016 & Ridlo W. Wibowo (ridlo.w.wibowo@gmail.com)\\
                   & Sulistiyowati (sulis.astro08@gmail.com)
\end{tabular}

\vspace{0.3cm}
\noindent
Solusi ini dibuat tanpa jaminan kesesuaian dengan solusi resmi dari juri olimpiade sains bidang Astronomi. Pengguna boleh menyebarluaskan dan/atau memodifikasi solusi ini dengan mencantumkan sumber asli. Hak cipta soal ada pada Kementerian Pendidikan dan Kebudayaan dan dilindungi undang-undang.

\vspace{0.4cm}
\noindent
\rule[2ex]{\textwidth}{1.5pt}

\textbf{Soal Pilihan Ganda}

\begin{questions}
\question Perbedaan utama antara lensa dan cermin sebagai pengumpul cahaya dalam sistem teleskop adalah
\begin{choices}
\choice Panjang fokus lensa bergantung pada panjang gelombang, sedangkan cermin tidak
\choice Panjang fokus cermin bergantung pada panjang gelombang, sedangkan lensa tidak
\choice Cermin berbentuk cembung, sedangkan lensa harus berbentuk cekung
\choice Objek tidak dapat ditegakkan dengan kedua sistem pengumpul cahaya ini
\choice Membuat cermin lebih mudah daripada membuat lensa untuk pengumpul cahaya
\end{choices}

\textit{Jawaban: A}\\



\vspace{0.5cm}
\question Pada tanggal 17 Januari 2016, komet Catalina berada pada posisi terdekatnya, yaitu 110 juta km dari Bumi. Jika pada saat itu terdapat ekor tampak sepanjang $1^{\circ}$ dan diasumsikan ekor tegak lurus garis pandang, maka panjang ekor (dalam satuan km) adalah
\begin{choices}
\choice $2 \times 10^6$
\choice $1 \times 10^6$
\choice $9 \times 10^7$
\choice $1 \times 10^8$
\choice $2 \times 10^8$
\end{choices}

\textit{Jawaban: }\\


\vspace{0.5cm}
\question Radius bintang katai putih memenuhi hubungan
\begin{equation*}
R \approx \frac{R_{\odot}}{74} \left( \frac{M_{\odot}}{M} \right)^{1/3}
\end{equation*}
Jika bintang katai putih dianggap sebagai benda hitam sempurna serta memiliki massa 0,4 massa Matahari ($M_{\odot}$) dan temperatur efektif ($T_{\text{eff}}$) 10000 K, maka luminositas bintang (dalam satuan $L_{\odot}$) adalah
\begin{choices}
\choice $3,6 \times 10^{-4}$
\choice $3,0 \times 10^{-3}$
\choice $1,8 \times 10^{2}$
\choice $2,6 \times 10^{3}$
\choice $7,6 \times 10^{4}$
\end{choices}

\textit{Jawaban: }\\


\vspace{0.5cm}
\question Misalkan $\mathcal{M}$ menyatakan magnitudo mutlak bolometrik sebuah bintang dan $\mathcal{P}$ menyatakan energi yang dipancarkan setiap detik (dalam satuan watt), maka berlaku hubungan
\begin{choices}
\choice $\mathcal{P} = 3,485 \times 10^{26} \times 10^{-0,4 \mathcal{M}}$
\choice $\mathcal{P} = 3,013 \times 10^{28} \times 10^{-0,4 \mathcal{M}}$
\choice $\log{\mathcal{P}} = 26,591 - 0,4 \mathcal{M}$
\choice $\log{\mathcal{P}} = 28,479 + 0,4 \mathcal{M}$
\choice $\mathcal{P} = -2,500 \log{\mathcal{M}}$
\end{choices}

\textit{Jawaban: }\\


\vspace{0.5cm}
\question Di antara lima pernyataan berikut, manakah pernyataan yang benar?  
\begin{choices}
\choice Urutan garis spektrum dalam kemunculam meraka pada bintang-bintang dengan menurunnya temperatur adalah: garis hidrogen yang sangat kuat, garis helium terionisasi, garis helium netral, garis logam netral, metal terionisasi, pita molekul titanium oksida
\choice Periode bintang ganda visual selalu lebih pendek daripada periode bintang ganda spektroskopi
\choice Bintang Sirius adalah bintang paling terang di langit, berarti ia merupakan bintang paling dekat dengan kita
\choice Bintang Alpha Centauri adalah bintang paling dekat dengan kita, berarti ia merupakan bintang paling terang di langit
\choice Sifat paling dasar dari bintang yang menentukan lokasinya di deret utama adalah massanya 
\end{choices}

\textit{Jawaban: E}\\


\vspace{0.5cm}
\question Perhatikan gambar di bawah yang merupakan skema kerja interferometer teleskop radio A dan B.


Untuk mencapai resolusi sudut 1" dari objek astronomi yang di amati pada panjang gelombang 21 cm orde pertama, maka jarak minimal antara teleskop radio A dan B adalah
\begin{choices}
\choice 40,8 km 
\choice 41,3 km
\choice 42,3 km
\choice 43,3 km
\choice 44,3 km
\end{choices}

\textit{Jawaban: }\\


\vspace{0.5cm}
\question Rasi bintang zodiak yang dapat diamati pada saat bersamaan dengan peristiwa Gerhana Matahari Total 2016  di Indonesia, adalah
\begin{choices}
\choice Pisces
\choice Gemini
\choice Virgo
\choice Orion
\choice Canis
\end{choices}

\textit{Jawaban: A}\\


\vspace{0.5cm}
\question Pada tanggal 9 Maret 2016 pagi hari akan berlangsung Gerhana Matahari Total yang jalurnya melewati Indonesia, di antaranya kota Palembang, Palangkaraya, dan Palu. Perkiraan posisi Bulan dalam sistem koordinat ekuatorial ($\alpha, \delta$) pada saat tersebut adalah
\begin{choices}
\choice (23 jam, $-4^{\circ}$)
\choice (01 jam, $+4^{\circ}$)
\choice (23 jam, $-1^{\circ}$)
\choice (23 jam, $+1^{\circ}$)
\choice (24 jam, $+0^{\circ}$)
\end{choices}

\textit{Jawaban: A}\\


\vspace{0.5cm}
\question Diketahui jarak rata-rata Bumi-Bulan adalah 384400 km dan periode orbit Bulan adalah 27,3 hari. Berapakah periode orbit sebuah satelit buatan yang mengitari Bumi pada ketinggian 96000 km jika orbitnya berupa lingkaran?
\begin{choices}
\choice 3,41 hari
\choice 3,76 hari
\choice 7,28 hari
\choice 10,40 hari
\choice 10,70 hari
\end{choices}

\textit{Jawaban: }\\



\vspace{1cm}
\textbf{Untuk empat soal berikut ini (No. 10\---13), jawablah}\\
A. jika 1, 2, dan 3 benar\\
B. jika 1 dan 3 benar\\
C. jika 2 dan 4 benar\\
D. jika 4 saja benar\\
E. jika semua benar\\

\vspace{0.5cm}
\question Jika atmosfer Bumi 50\% lebih rapat daripada keadaan saat ini, maka
\begin{enumerate}
\item Cahaya Matahari akan tampak lebih merah daripada keadaan sekarang, karena dengan bertambahnya kerapatan, akan lebih banyak cahaya pada panjang gelombang biru yang dihamburkan ke segala arah.
\item Cahaya yang sampai ke permukaan Bumi akan semakin tampak berwarna biru 
\item Cahaya yang sampai ke permukaan Bumi akan semakin tampak berwarna merah 
\item Cahaya Matahari akan tampak lebih biru daripada keadaan sekarang, karena dengan bertambahnya kerapatan, akan lebih banyak cahaya pada panjang gelombang merah yang dihamburkan ke segala arah.
\end{enumerate}

\textit{Jawaban: }\\
Penjelasan untuk masing-masing pernyataan:
\begin{enumerate}
\item 
\item 
\item 
\item 
\end{enumerate}


\vspace{0.5cm}
\question Konstanta Hubble $H_0$ menyatakan laju pengembangan alam semesta saat ini. Karena kesalahan dalam penentuan jarak beberapa galaksi yang digunakannya, Edwin Hubble mendapatkan nilai sekitar 500 km/s/Mpc untuk $H_0$. Nilai yang diterima sekarang oleh para astronom berdasarkan berbagai pengukuran adalah 70 km/s/Mpc. Manakah pernyataan-pernyataan yang benar di bawah ini?
\begin{enumerate}
\item Laju pengembangan alam semesta menurut Hubble lebih cepat daripada seharusnya
\item Jarak galaksi-galaksi yang digunakan Hubble lebih jauh daripada seharusnya
\item Umur alam semesta menurut Hubble lebih muda daripada umur sebenarnya
\item Jika sejak \textit{Big Bang} alam semesta mengembang dengan laju konstan $H_0$, maka menurut Hubble ukuran alam semesta saat ini kurang lebih sepertujuh dari nilai yang diterima sekarang
\end{enumerate}

\textit{Jawaban: }\\
Penjelasan untuk masing-masing pernyataan:
\begin{enumerate}
\item 
\item 
\item 
\item 
\end{enumerate}


\vspace{0.5cm}
\question Di antara empat pernyataan berikut, yang merupakan karakteristik materi antar bintang adalah
\begin{enumerate}
\item Dalam besaran massa, materi antar bintang tersusun atas Hidrogen, Helium, dan sedikit unsur berat 
\item Daerah hidrogen terionisasi (\textit{HII region}) terjadi akibat radiasi ultraviolet dari bintang panas di dekatnya
\item Materi antar bintang dapat terkait dengan pembentukan dan kematian bintang
\item Hanya ada satu kelompok daerah dalam materi antar bintang yaitu yang berhubungan dengan pembentukan bintang
\end{enumerate}

\textit{Jawaban: }\\
Penjelasan untuk masing-masing pernyataan:
\begin{enumerate}
\item 
\item 
\item 
\item 
\end{enumerate}


\vspace{0.5cm}
\question Yang merupakan karakteristik dalam proses pembentukan bintang adalah
\begin{enumerate}
\item Materi antar bintang yang melimpah dalam periode waktu yang lama, tidak selalu cukup untuk membentuk bintang
\item Proses pembentukan bintang dapat terjadi bila energi kinetik materi antar bintang lebih kecil dari setengah energi potensial gravitasi
\item Syarat terjadinya pembentukan bintang bergantung pada temperatur dan kerapatan partikel
\item Proses pembentukan bintang hanya dapat terjadi di daerah dengan konsentrasi debu tinggi
\end{enumerate}

\textit{Jawaban: }\\
Penjelasan untuk masing-masing pernyataan:
\begin{enumerate}
\item 
\item 
\item 
\item 
\end{enumerate}


\vspace{1cm}
\textbf{Gunakan petunjuk ini untuk menjawab dua soal berikut (No. 14-15):}\\
A. Pernyataan pertama dan kedua benar serta memiliki hubungan sebab-akibat.\\
B. Pernyataan pertama dan kedua benar, tetapi tidak memiliki hubungan sebab-akibat.\\
C. Pernyataan pertama benar, sedangkan pernyataan kedua salah.\\
D. Pernyataan pertama salah, sedangkan pernyataan kedua benar\\
E. Kedua pernyataan salah.\\

\vspace{0.5cm}
\question Tepi bayangan sebuah objek yang dibentuk oleh sinar Matahari tidak tajam
\begin{center}
SEBAB
\end{center}
\noindent Jarak Bumi dari Matahari yang sangat jauh menyebabkan Matahari dianggap sebagai suatu sumber titik dan cahayanya mengalami difraksi atau pembelokan cahaya  

\textit{Jawaban: }\\


\vspace{0.5cm}
\question Bulan mengorbit Bumi, sedangkan asteroid mengorbit Matahari.
\begin{center}
SEBAB
\end{center}
\noindent Ukuran asteroid lebih kecil daripada Bulan.

\textit{Jawaban: }\\


\vspace{1cm}
\textbf{Soal Isian Singkat}

\vspace{0.5cm} 
\question Jarak antara dua sumber titik yang direkam oleh detektor pada bidang fokus sebuah teleskop bergantung pada $\ldots\ldots\ldots$

\textit{Jawaban: }\\
Panjang fokus teleskop dan jarak pisah dua titik (jarak sudut).


\vspace{0.5cm}
\question Gugus bola IAU C0923 545 memiliki magnitudo semu $V = +13,0$ dan magnitudo mutlak $M_V = -4,15$. Gugus ini terletak 9,0 kpc dari Bumi, 11,9 kpc dari pusat Galaksi, dan sekitar 0,5 kpc di selatan bidang Galaksi. Besar serapan materi antar bintang per kpc yang kita amati ke arah IAU C0923 545 adalah $\ldots\ldots\ldots$

\textit{Jawaban: }\\


\vspace{0.5cm}
\question Diduga terdapat sebuah planet X yang merupakan planet kesembilan di Tata Surya. Planet yang diperkirakan seukuran Uranus ini belum pernah teramati karena jaraknya yang jauh (20 kali jarak Matahari-Neptunus). \textit{Very Large Telescope} digunakan untuk mencari keberadaan planet tersebut dengan menggunakan teknik interferometer. Dengan detektor inframerah 20 mikron, jarak pisah minimal antar teleskop (\textit{baseline}) yang dibutuhkan untuk dapat menentukan ukuran planet secara langsung adalah sekitar $\ldots\ldots\ldots$

\textit{Jawaban: }\\


\vspace{0.5cm}
\question Salah satu sumber pemancar radio terkuat di langit setelah Matahari dan Cassiopeia A (sisa supernova yang relatif dekat) adalah Galaksi Cygnus A. Pada panjang gelombang 0,75 cm, rapat fluks spektral yang diukur oleh teleskop radio dengan diameter 25 m adalah 4500 Jy (jansky). Dengan menganggap efisiensi teleskop radio 100\% dan lebar pita frekuensi adalah 5 MHz, daya total yang dideteksi penerima adalah sebesar $\ldots\ldots\ldots$ watt.  

\textit{Jawaban: }\\


\vspace{0.5cm}
\question Jika Matahari berevolusi menjadi bintang raksasa merah dengan ukuran 100 kali jejari Matahari saat ini dan temperaturnya menjadi 3200 $^{\circ}$C, maka magnitudo semu Matahari raksasa adalah $\ldots\ldots\ldots$

\textit{Jawaban: }\\


\vspace{1cm}
\textbf{Soal Esai}

\vspace{0.5cm}
\question Pada pukul 18:00, tinggi dan azimuth Bulan adalah $+39^{\circ}$ dan $196^{\circ}$, sedangkan tinggi dan azimuth Saturnus adalah $+34^{\circ}$ dan $210^{\circ}$. Jika Bulan dan Saturnus dianggap benda titik, hitunglah jarak pisah kedua benda tersebut dalam satuan derajat!  

\textit{Jawaban: }\\


\vspace{0.5cm}
\question Dengan teleskop Hubble, astronom dapat mengamati bintang seperti Matahari pada jarak 100 kpc. Bintang-bintang Cepheid yang paling terang memiliki kecerlangan intrinsik 30000 kali lebih besar daripada Matahari. Jika serapan materi antar bintang diabaikan, tentukanlah jarak Cepheid terjauh yang dapat diamati teleskop Hubble!

\textit{Jawaban: }\\


\vspace{0.5cm}
\question Gunung Olympus yang memiliki tinggi 25 km adalah gunung tertinggi di Mars sekaligus tertinggi di Tata Surya. Bahkan tiga kali lebih tinggi dari gunung Himalaya. Untuk mengetahui mengapa planet Mars dapat memiliki gunung tertinggi di Tata Surya, maka
\begin{enumerate}[(a)]
\item Hitunglah tinggi maksimum gunung di Bumi bila gunung dianggap berbentuk kerucut! Ambillah nilai batas elastisitas kerak Bumi sebesar $3 \times 10^8$ N/m$^{2}$!
\item Jelaskan mengapa gunung tertinggi di Tata Surya dapat berada di planet Mars!
\end{enumerate} 

\textit{Jawaban: }\\


\vspace{0.5cm}
\question Di Galaksi, sebuah proton berenergi $10^7$ eV melintasi medan magnet antar bintang yang homogen ($B = 3 \times 10^{-10}$ T) dalam arah tegak lurus sehingga bergerak melingkar. Proton tersebut memiliki laju awal $v \ll c$ (non relativistik).
\begin{enumerate}[(a)]
\item Hitunglah momentum linear proton tersebut!
\item Hitunglah radius gerak melingkar proton (\textit{gyroradius}) dan frekuensi sudut yang dihasilkan!
\item Gambarkan arah gerak dan lintasan proton jika proton membentuk sudut tertentu (dengan sudut sekitar $30^{\circ}$) terhadap garis medan magnet!
\end{enumerate}

\textit{Jawaban: }\\


\vspace{0.5cm}
\question Pada akhir Januari 2016 terjadi fenomena langka yaitu 5 planet klasik tampak berparade di langit fajar. Merkurius tampak di arah Timur, diikuti Venus, Saturnus, Mars, dan Jupiter di sebelah Barat. Jika jarak sudut Jupiter dan Merkurius adalah $100^{\circ}$ sementara sudut elongasi Merkurius dari Matahari adalah $15^{\circ}$, hitunglah jarak antara Jupiter dan Merkurius dalam satuan astronomi! Anggap orbit kelima planet berada dalam satu bidang!

\textit{Jawaban: }\\



\end{questions}

\vspace{1cm}
\begin{flushright}
Dapat diperoleh di \href{http://ridlow.wordpress.com}{http://ridlow.wordpress.com}
\end{flushright}
\end{document}
